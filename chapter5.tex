\label{chapter5}


\chapter{Conclusion and Recommendations}
\section{Introduction}
This chapter discusses the conclusions reached as a result of the research. Also included in this chapter are recommendations related to the research project.



\section{Conclusions}
The purpose of this study was to determine the influence of climate change on malaria vectorial capability across Ghana's diverse agro-ecological zones. The extrinsic incubation duration as well as the survival probability across the various agro-ecological zones were estimated using the average monthly temperature from 1981 to 2020.
The vectorial capacity increases as the temperatures increases for maximum and minimum temperature range and decreases with decreasing temperatures for minimum temperatures due to the fact that at extreme low temperatures the digestion rate of the vectors is low. However, for the survival probability, increasing in temperature decreases the probability of the malaria vector to survive and vice-versa.
The trend of the graph revealed that, at rainy/wet seasons the vectorial capacity together with the survival probability were higher than during dry seasons.\\

\noindent The quest to find out whether vectorial capacity of the malaria vector differ as a function of climate and its environment in Ghana was answered by the study reviewing that at the northern zone the vectorial capacity and survival probability were higher during rainy season but lower during dry season whiles for coastal and the forest zones the vectorial capacity and survival probability of the vectors were higher throughout the year.\\

\noindent Having knowledge of how climate change influence the vectorial capacity of the malaria vector over their agro-ecological zones of Ghana will be of great importance to the country and will result in many benefits. Having such kind of information will provide research information on how the warming climate influence the capacity of malaria vectors. Also, it will provide knowledge on the differential burden of malaria as a function of climate and environment.
As a country, we face an inevitable challenge pertaining to the high mortality
rate especially in infants, which threatens the future of Ghana. As a matter of fact, this study will provide information useful for the design of malaria control programs and possible elimination of the disease.         
With this knowledge, targetted interventions can be made to counter the mortality and morbidity of malaria in Ghana especially in infants. The outcome of this study would be of much importance in contributing to achieving the Sustainable Development Goal (SDG) number 3.

\newpage
\section{Recommendations from this study}
This study recommends that: 
\begin{enumerate}
\item Government and stakeholders in controlling of malaria should target the northern zone during the rainy season and target both the coastal and the forest zone throughout the year.

\item Other future works should focus on more cities across the various agro-ecological zones since each city have it own micro-climate which influence the malaria vectors.

\end{enumerate}





