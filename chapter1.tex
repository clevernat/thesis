\label{chapter1}

\chapter{Introduction}
\section{Background to the study}
Malaria is a deadly parasitic disease that is spread to humans via female Anopheles mosquito bites \citep{world2014malaria}.It's both avoidable and treatable. Plasmodium parasites are the cause of malaria.
The five malaria parasites are Plasmodium vivax, Plasmodium falciparum, Plasmodium malariae, Plasmodium ovale, and Plasmodium knowlesi burden. Of these five known malaria parasites, it is the P. falciparum that causes most of the malaria burden globally. Malaria is not always caused by all five parasites in all parts of the world. Plasmodium falciparum, Plasmodium malariae, and Plasmodium ovale are the pathogens in Ghana, for example. The main vectors of these parasites are Anopheles gambiae s.s. and Anopheles funestus s.s.\\

\noindent According to \cite{world2020malaria}, there were an expected 229 million cases of malaria globally, with an estimated 409,000 malaria deaths in 2019. The most vulnerable to malaria were children under the age of five, who accounted for 67 percent (274 000) of all malaria deaths worldwide in 2019. The disease is predominantly a tropical disease with a high morbidity and mortality rate, as well as a significant economic and social burden.
It is a worldwide public health problem. About 90\% of the malaria episodes occurred in the WHO Africa. This has been the trend over the years.
Only 15 out of all the countries in the world accounts for 80\% of these malaria cases. Of these, 14 are in the sub-Saharan Africa (SSA)
Thus, there is the need for most of the efforts and support for malaria control to be directed to the heavily affected places in the SSA. This is necessary to enable the global malaria response to get back on track.
Ghana is one of the SSA countries responsible for 80\% of the global cases.\\

\noindent Climate change has an impact on the malaria population and vector density. Climate change will increase the likelihood of malaria transmission in both traditionally malarious and suppressed areas, as well as previously non-malarious places. Increases in temperature and rainfall may encourage the spread of malaria-carrying mosquitoes, resulting in an increase in malaria transmission. Rising temperatures will disrupt the parasite's growth cycle in the mosquito, allowing it to develop faster and so influencing disease load at lower elevations where malaria is already a problem. The World Health Organization and the World Meteorological Organization have classed malaria as one of the most climate-sensitive diseases, with a plethora of evidence suggesting significant correlations between temperature, rainfall, and humidity and malaria occurrence.\\


 \noindent Over an eight-year period (2001-2008), \cite{klutse2014assessment} examined climatic data and malaria cases from two ecological zones in Ghana (the transition and coastal savannah zones, respectively) to see if there was a link between malaria cases, temperature and rainfall patterns, and the potential effects of climate change on malaria epidemiological trends. The data suggest that, especially in the transition zone, maximum temperature, rather than minimum temperature or precipitation, is a better predictor of malaria trends.

\newpage
\noindent The purpose of this research is to see how climate change affects malaria vectorial competence in Ghana's various agro-ecological zones. Effect of the climatic factors such as temperature will be the main focus in this study. Using temperature related functions and temperature data from the Ghana Meteorological Agency, we will investigate how seasonal temperature fluctuations influenced the capability and survival of malaria vectors across Ghana's agro-ecological zones. 

\section{Statement of the problem}
Malaria transmission intensity is greatly reliant on local mosquito vectorial capability and competency \citep{cohuet2010evolutionary}. As a result, the vectorial capability of malaria vectors is affected by climatic change. In recent years, several studies on the effects of climate change on malaria vectors have made tremendous progress; however, some aspects have been missed, and the impact of climate change on malaria vector vectorial ability is commonly overlooked.\\

\noindent \cite{martens1995potential}, for example, worked on a study called Global Climate Change and Malaria Risk. They arrived to the conclusion that changes in malaria transmission linked to socioeconomic development, population expansion, and the efficacy of control efforts will all influence the degree of a rise in malaria risk.\\

 \noindent In Ghana's two ecological zones, \cite{klutse2014assessment} looked at the links between climate variables and malaria cases.The maximum temperature, rather than the lowest temperature or precipitation, was found to be a stronger predictor of malaria trends, especially in the transition zone.

\newpage
\noindent \cite{mohammadkhani2016relation} conducted study in Kerman, Iran, on the relationship between meteorological conditions and malaria incidence. Temperature was discovered to be the most effective meteorological element in the occurrence of malaria. The incidence rate grew dramatically as the mean maximum and minimum monthly temperatures increased. Finally, temperature is one of the most important climate variables influencing the occurrence of malaria, and it should be taken into account while planning for disease control and prevention.\\


\noindent \cite{christian2021effect} looked on the impact of malaria adaptive ability on subjective well-being in Ghana. To summarize, households receiving social assistance demand a lower amount of income. Those who provide support and those who do not provide any support are compared to achieve the same degree of verbal quantification of welfare.\\


\noindent Also one of the research work which focused on the vectorial capacity of the malaria vector had little information and it was concentrated at the epidemic region of Africa that article was conducted by \citep{ceccato2012vectorial}. They used a tool termed vectorial capacity to track changes in malaria transmission potential in epidemic locations across Africa. The vectorial capacity model (VCAP) was expanded in this study to account for the impact of rainfall and temperature factors on malaria transmission potential. They merged data from two remote sensing devices to monitor rainfall and temperature, and the VCAP was developed. They discovered that Eritrea has a robust malaria control program, with malaria incidence dropping significantly between 2000 and 2009. Malaria outbreaks are still a possibility in the country due to changing weather patterns. In conclusion, the extended VCAP accurately tracks the risk of malaria in both rainy and non-rainy locations.

\newpage
\noindent Although the above research articles provide information on how climate change affect malaria vectors in many aspects. They mostly have limited information on how climate factors impact on the vectorial capacity or the population density of these malaria vectors. Also, most of these works conducted about or in Ghana was centered on one of the agro-ecological zones. Therefore there is the need to research in these areas since different agro-ecological zones have different climatic conditions. This research will help researchers better understand how climate change affects the capacity of malaria vectors, as well as provide information for the development of malaria control programs and the eventual eradication of the disease.

\section{Study Question}
The study asked the following questions to discover solutions or answers to the outline difficulties.

\begin{enumerate}
	
	\item How does seasonal changes in temperature influence the seasonality of the survival and vectorial capacity of malaria vectors over the agro-ecological zones of Ghana?
	\item Does survival probability and vectorial capacity of malaria vectors differ from one agro-ecological zone to the other?

\end{enumerate}

\section{Objectives}
To assess the impact of climate change on the vectorial capacity and survival probability of malaria vectors over the agro-ecological zones of Ghana.\\
Specific objectives include:\\
\begin{enumerate}
	
	\item Determine the influence of seasonal  temperature  change on the survival and malaria transmission capacity of malaria vectors.
	
	
	\item  Determine whether  the survival and vectorial capacity of malaria vectors differ as a function of climate and environment in Ghana.
	
	
\end{enumerate}

\section{Research significance and justification}
Having knowledge of how climate change influence the vectorial capacity and survival probability of the malaria vectors over their agro-ecological zones of Ghana will be of great importance to the country and will result in many benefits. Having such kind of information will provide research information on how the warming climate influence the capacity of malaria vectors. Also, it will provide knowledge on the differential burden of malaria as a function of climate and environment.
 As a country, we face an inevitable challenge pertaining to the high mortality
 rate especially in infants, which threatens the future of Ghana. As a matter of fact, this study will provide information useful for the design of malaria control programs and possible elimination of the disease.         
With this knowledge, targetted interventions can be made to counter the mortality and morbidity of malaria in Ghana especially in infants. The outcome of this study would be of much importance in contributing to achieving the Sustainable Development Goal (SDG) number 3.
\newpage

\section{Organization of the thesis}
The thesis is divided into five sections. The first chapter introduces the project ideas, which include the research backdrop, problem statement, objectives, and research justification. A review of prior efforts is discussed in chapter two. It starts with a basic overview of malaria and the parasite that causes the disease. The review also looked at the malaria life cycle and how climatic and non-climatic factors affect malaria transmission. The study area, data, and method used in this project are all covered in the third chapter of this paper. Chapter 4 covers the results and discussion, while Chapter 5 covers the conclusion, recommendations, and references.